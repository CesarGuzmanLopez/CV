\begin{rubric}{Experiencia}
	\subrubric{Cursos}
    \entry*[] \textbf{Curso de Programación "Introducción en C++"} en la Universidad Autónoma Metropolitana Unidad Iztapalapa (2019).
    \entry*[] \textbf{Curso de Punteros y Memoria Dinámica en C++} en la Universidad Autónoma Metropolitana Unidad Iztapalapa (2019).
    \entry*[] \textbf{Tutorías de Programación Web y Desarrollo de Aplicaciones Móviles con Tecnologías Web} (2021).
    
	\subrubric{Proyectos}
    \entry*[] \textbf{Proyecto de Programación en C++} en la Universidad Autónoma Metropolitana Unidad Iztapalapa (2020).
       Traducción de programas hechos en Fortran:
       \begin{itemize}
           \item Molar Interactions with Parameters
           \item Marcus Kinetics
       \end{itemize}
    \entry*[] \textbf{Proyecto de Programación en C\#} en la Universidad Autónoma Metropolitana Unidad Iztapalapa (2017).
       Desarrollo de aplicaciones en C\# con interfaces gráficas:
       \begin{itemize}
           \item Molar Fractions
           \item MarcusKin EasyRate
           \item Disponible en: \url{https://agalano.com/apps/desktop-apps/}
       \end{itemize}
    \entry*[] \textbf{Proyecto de Programación en Laravel}.
       Desarrollo de base de datos de antioxidantes y sus propiedades moleculares (Molecular Properties Database).
       \begin{itemize}
           \item Generación de gráficas y relaciones de componentes.
           \item Disponible en: \url{http://chemistry.agalano.com/Kinetics/relative-k-overall}
       \end{itemize}
    \entry*[] Desarrollo y mantenimiento de la página web de la Sociedad Química Mexicana.
       \begin{itemize}
           \item Desarrollo y puesta a punto de tabla periódica digital.
           \item Generación de formularios de registro y pago de membresías.
           \item Disponible en: \url{https://sqm.org.mx/}
       \end{itemize}
    \entry*[] Desarrollo de aplicaciones de dinámica molecular y proceso CADMA-G.
       \begin{itemize}
           \item Disponible en: \url{https://github.com/CesarGuzmanLopez/Cadma-G}
       \end{itemize}
    \entry*[] Desarrollo de paquetería en 2022 para leer archivo de salida de los logs de Gaussian en Python.
       \begin{itemize}
           \item Disponible en: \url{https://github.com/CesarGuzmanLopez/read_log_gaussian}
       \end{itemize}
    \entry*[] Desarrollo y asesoría en 2021 de la página de la Editorial Aquitania Siglo 21.
       \begin{itemize}
           \item Disponible en: \url{https://aquitania-xxi.com/}
       \end{itemize}
    \entry*[] Desarrollo de página para el 12° Taller de Dinámica Molecular y el 13th Meeting on Molecular Simulations en 2023.
       \begin{itemize}
           \item Disponible en: \url{https://simulacion-molecular.mx/}
       \end{itemize}
    \entry*[] Desarrollador oficial de diversos proyectos de software libre.
\end{rubric}
